\documentclass[usenatbib]{mnras}

\usepackage[varg]{newtxmath}
\usepackage{newtxtext}
\usepackage[utf8]{inputenc}
\usepackage{graphicx}
\usepackage{microtype}
\usepackage{xcolor}
\usepackage{fixltx2e}
\usepackage{booktabs}
\usepackage{hyperref}
\usepackage{siunitx}
\usepackage{color}
\hypersetup{colorlinks=True, linkcolor=blue!50!black, citecolor=black,
  urlcolor=blue!50!black}
\usepackage{microtype}

\graphicspath{ {../}, }

\title[Will's extra material]{Extra knot material from Will for the
  Alma paper}

\newcommand\AddressIRyA{Instituto de Radioastronom\'{\i}a y Astrof\'{\i}sica,
  Universidad Nacional Aut\'onoma de M\'exico, Apartado Postal 3-72,
  58090 Morelia, Michoac\'an, M\'exico}

\newcommand\AddressEnsenada{Instituto de Astronom\'{\i}a, Universidad
  Nacional Aut\'onoma de M\'exico, Km 103 Carretera Tijuana-Ensenada,
  22860 Ensenada, Baja California, México}

\author[Fernández-Martín et al.]{
  Alba Fernández-Martín,\textsuperscript{1}
  William J. Henney,\textsuperscript{1}
  M. Teresa García-Díaz,\textsuperscript{2}
  \& S. Jane Arthur\textsuperscript{1}\\
  \textsuperscript{1}\AddressIRyA\\
  \textsuperscript{2}\AddressEnsenada\\
}
\date{Accepted XXX. Received YYY; in original form ZZZ}

\pubyear{2017}
\begin{document}
\label{firstpage}
\pagerange{\pageref{firstpage}--\pageref{lastpage}}
\maketitle

\begin{abstract}
New material written by Will in 2016 December, describing methodology,
results, and interpretation from new knot measurements and fitting.
\end{abstract}

% Select between one and six entries from the list of approved keywords.
% Don't make up new ones.
\begin{keywords}
keyword1 -- keyword2 -- keyword3
\end{keywords}

\newcommand\nii{\ensuremath{\ion{N}{ii}}}
\newcommand\ha{\ensuremath{\mathrm{H\alpha}}}

\section{Knot classification}
\label{sec:knot-classification}

\section{Knot Analysis}
\label{sec:knot-analysis}

\begin{figure}
  \centering
  \includegraphics[width=\linewidth]{knot-dv-versus-nii-ha-ratio}
  \caption{Correlation between [\nii]--\ha{} velocity difference,
    \(\Delta V\), versus line ratio, \(R_{[\nii]}\), for different
    datasets. The grayscale cloud shows the inner Huygens region of
    the nebula, with data from the MUSE spectra of \citet{MUSE} and
    orange dashed line indicating the trend line (obtained by
    averaging the \(\Delta V\) values within \(R_{[\nii]}\) bins of
    width 0.01).  }
\end{figure}

\end{document}

%%% Local Variables:
%%% mode: latex
%%% TeX-master: t
%%% End:
